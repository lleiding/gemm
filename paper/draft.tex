\documentclass[a4paper]{scrartcl}
\usepackage[english]{babel}
\usepackage{natbib}
\usepackage[T1]{fontenc}
\usepackage[utf8]{inputenc}

\title{The relative roles of genetic, ecological and environmental factors in species radiations}
\author{Ludwig Leidinger}
\date{\today}
%
\begin{document}
\maketitle

\begin{abstract}

\end{abstract}

\section{Introduction}
island biota have always fascinated and inspired researchers (Darwin, Wallace, ...)

This interest in islands and their recognition as highly suitable model systems resulted in the formulation of
important theoretical frameworks, foremost MacArthur and Wilson's seminal equilibrium theory of island biogeography (ETIB).
Later ... GDM.

These theories have in common a focus on species numbers and respective rates of migration, extinction and speciation.
While they are successful in describing species numbers, they are neutral in design (in the UNTB sense) and thus were not meant to consider species traits and compositions.

Yet, there are numerous examples for fascinating patterns of trait distributions and evolution on islands, such as island syndromes[, ...] or the fact that some lineages radiate explosively following
island colonisation, while others remain one single species.

These patterns have been explained theoretically [here, here and here], but usually one at a time.
In the following, we aim to produce and thus explain these forementioned patterns using a single framework, and thus following the same set of (rather simple) rules.

Using an individual-based mechanistic model we investigated how genetic, ecological and environmental factors colonisation and radiation histories on islands,
and what trait combinations constitute successful colonizer and radiator [this sounds like heating ;)] species, respectively.
\section{Material and Methods}

\section{Results}

\section{Discussion}

\addbibresource{refs.bib}
\printbibliography

\end{document}
